\section*{Login and running}

\begin{frame}[fragile]
  \frametitle{Hands-on exercise}

\footnotesize
\begin{exampleblock}{\large{Login}}
\begin{itemize}
  \item Some configuration steps are needed to log in to PDC 
  \item Depends on OS: https://www.pdc.kth.se/support/documents/login/login.html
  \item In short, Kerberos and SSH supporting GSSAPI key exchange must be installed 
  \item If needed, you will receive help to connect from your own laptops
\end{itemize}
\end{exampleblock}

\begin{exampleblock}{\large{Exercises}}
\begin{itemize}
  \item You will now practice key steps in using PDC resources
  \item The example source code and batch scripts can be found at \verb|/afs/pdc.kth.se/home/k/kthw/Public/pdc_test.tar.gz|,
    or in the GitHub repository \verb|https://github.com/PDC-support/introduction-to-pdc| (in the intro-course branch)
\end{itemize}
\end{exampleblock}

\end{frame}


%-------------------------------------------------------------------------------------

\begin{frame}[fragile]
  \frametitle{Hands-on exercise}

\footnotesize

\begin{exampleblock}{\large{Exercises}}
The exercises below will demonstrate:
\begin{enumerate}
  \item Kerberos - creating and listing tickets
  \item Login - via command line or PuTTY, and where you end up on PDC
  \item AFS home, information on your projects, \verb|/cfs/klemming|
  \item Working with modules
  \item Compiling code
  \item Submitting jobs
  \item Running interactively
\end{enumerate}
\end{exampleblock}

\begin{exampleblock}{\large{Further information}}
\begin{itemize}
  \item https://www.pdc.kth.se/support
  \item https://hpc-carpentry.github.io/hpc-intro/
  \item https://software-carpentry.org/lessons/
\end{itemize}
\end{exampleblock}

\end{frame}


%-------------------------------------------------------------------------------------


\begin{frame}[fragile]
  \frametitle{Kerberos}
\begin{itemize}
  \item Two files should be created/edited to enable login to PDC, \verb|krb5.conf| and (linux and mac) \verb|.ssh/config|.
    For Windows, SSH configuration is done in PuTTY. 
    Further information here: https://www.pdc.kth.se/support/documents/login/configuration.html

  \item Assuming that you have now configured Kerberos and SSH, create a forwardable Kerberos ticket
  \item List the ticket. Can you see if it is forwardable? What does all the jargon mean?

\end{itemize}
\end{frame}

\begin{frame}[fragile]
  \frametitle{Kerberos}
  \begin{exampleblock}{Answer}
    \verbatimfont{\footnotesize}
    \begin{itemize}
    \item To create a forwardable Kerberos ticket, type:
    \begin{verbatim}
      $ kinit -f <username>@NADA.KTH.SE
    \end{verbatim}

  \item To list your Kerberos tickets, type:
    \begin{verbatim}
      $ klist -f
    \end{verbatim}
    The output will hopefully look similar to
   \begin{verbatim}
Credentials cache: FILE:/tmp/krb5cc_H26527
    Principal: kthw@NADA.KTH.SE
Issued                Expires             Flags    Principal
Aug  3 16:41:51 2017  Aug  4 16:39:50 2017  FfA    krbtgt/NADA.KTH.SE@NADA.KTH.SE
Aug  3 16:41:52 2017  Aug  4 16:39:50 2017  fA     afs/pdc.kth.se@NADA.KTH.SE
   \end{verbatim}
The first line is your Kerberos ticket, the second line your AFS token.
\end{itemize}
\end{exampleblock}

\end{frame}




%-------------------------------------------------------------------------------------
\begin{frame}[fragile]
  \frametitle{Login}
\begin{itemize}
  \item Start by logging in to Beskow
  \item Where are you after login?
    \begin{itemize}
      \item What is your current directory?
      \item What's the name of the login node?
    \end{itemize}
  \item Have a look at the currently running processes on the login node
\end{itemize}
\end{frame}


%-------------------------------------------------------------------------------------
\begin{frame}[fragile]
  \frametitle{Login}
\begin{exampleblock}{{Answer}}
    \verbatimfont{\footnotesize}
    \begin{itemize}
    \item On Linux/MacOS, type one of these (use PuTTY on Windows):
    \begin{verbatim}
    $ ssh username@beskow.pdc.kth.se
    $ ssh -o GSSAPIDelegateCredentials=yes -o GSSAPIKeyExchange=yes 
      -o GSSAPIAuthentication=yes username@beskow.pdc.kth.se
    \end{verbatim}

    \item After logging in, \verb|pwd| shows your current directory:
    \begin{verbatim}
    $ pwd
    /afs/pdc.kth.se/home/u/username
    \end{verbatim}

    \item The \verb|hostname| command shows the hostname of the login node:
    \begin{verbatim}
    $ hostname
    beskow-login2.pdc.kth.se
    \end{verbatim}

    \item The \verb|top| command shows a snapshot of currently running processes:
    \begin{verbatim}
      $ top
    \end{verbatim}

    \item Note that the login node is a shared resource!
    \end{itemize}

\end{exampleblock}
\end{frame}


%-------------------------------------------------------------------------------------
\begin{frame}[fragile]
  \frametitle{AFS home, listing projects, /cfs/klemming}
\begin{itemize}
  \item Check your AFS disk quota (hint: you will need the \verb|fs| command, type \verb|fs help| to see available subcommands)
  \item Go to your klemming nobackup directory. Can you run \verb|fs| there?
  \item Check which allocation(s) you belong to using the \verb|projinfo| command
    \begin{itemize}
      \item Check all options of \verb|projinfo| using the \verb|-h| flag
      \item How much have your allocations been used, and for how long are they valid?
    \end{itemize}
\end{itemize}
\end{frame}


%-------------------------------------------------------------------------------------
\begin{frame}[fragile]
  \frametitle{AFS home, listing projects, /cfs/klemming}
\begin{exampleblock}{{Answer}}
    \verbatimfont{\footnotesize}
    \begin{itemize}
    \item The \verb|fs lq| command shows the name of the AFS volume of your home directory, your disk quota and usage:
    \begin{verbatim}
    $ fs lq
    \end{verbatim}

    \item \verb|fs lq| only works on AFS. You can type \verb|df -h .| instead, but there's no quota on your klemming usage:
    \begin{verbatim}
    $ df -h .
    Filesystem     Size  Used Avail Use% Mounted on
    ...:/klemming  5.2P  4.4P  730T  86% /cfs/klemming
    \end{verbatim}

    \item The \verb|projinfo| command accesses a database that contains logs of all allocations 
    \begin{verbatim}
    $ projinfo
    \end{verbatim}

    \end{itemize}

\end{exampleblock}
\end{frame}




%-------------------------------------------------------------------------------------
\begin{frame}[fragile]
  \frametitle{Working with modules}
\begin{itemize}
  \item List the modules that are currently loaded (hint: type \verb|module help|)
  \item List all the modules that are available on Beskow
  \item List all available modules named \verb|PrgEnv|
  \item Swap from the \verb|PrgEnv-cray| to the \verb|PrgEnv-gnu| module
\end{itemize}
\end{frame}


%-------------------------------------------------------------------------------------
\begin{frame}[fragile]
  \frametitle{Working with  modules}
\begin{exampleblock}{{Answer}}
    \verbatimfont{\footnotesize}
    \begin{itemize}
    \item Listing all loaded modules and all available modules is done like this:
    \begin{verbatim}
      $ module list
      $ module avail
    \end{verbatim}

    \item To list all modules matching a pattern (like PrgEnv), type
    \begin{verbatim}
      $ module avail PrgEnv
    \end{verbatim}

    \item To swap programming environments (i.e. compiler environments), type
    \begin{verbatim}
      $ module swap PrgEnv-cray PrgEnv-gnu
    \end{verbatim}
    After swapping these modules, the compiler wrappers \verb|CC, cc| and \verb|ftn| point to the GNU compilers (instead of the 
    Cray compilers)

    \end{itemize}

\end{exampleblock}
\end{frame}

%-------------------------------------------------------------------------------------
\begin{frame}[fragile]
  \frametitle{Compiling code}
\begin{itemize}
  \item Copy the tarball \verb|/afs/pdc.kth.se/home/k/kthw/Public/pdc_test.tar.gz| to  
    your nobackup directory, and unpack it
  \item Now compile the MPI example code \verb|hello_world_mpi.c|
    \begin{itemize}
      \item Do you need to load any MPI libraries?
      \item What compiler will be used when using the \verb|cc| compiler wrapper?
    \end{itemize}
\end{itemize}
\end{frame}


%-------------------------------------------------------------------------------------
\begin{frame}[fragile]
  \frametitle{Compiling code}
\begin{exampleblock}{{Answer}}
    \verbatimfont{\footnotesize}
    \begin{itemize}
    \item Copying and extracting the tarball to your klemming directory:
    \begin{verbatim}
      $ cd /cfs/klemming/nobackup/u/username # replace this!
      $ cp /afs/pdc.kth.se/home/k/kthw/Public/pdc_test.tar.gz .
      $ tar zxf pdc_test.tar.gz
      $ cd pdc_test
    \end{verbatim}

    \item When using the compiler wrappers CC, cc and ftn, no MPI or numerical libraries need to be loaded 
      or linked to explicitly!
    \begin{verbatim}
      $ cc -o hello_world_mpi hello_world_mpi.c
    \end{verbatim}
    \item Which compiler did we just compile with?
    \begin{verbatim}
      $ cc --version
      gcc (GCC) 4.9.1 20140716 (Cray Inc.)
    \end{verbatim}

    \end{itemize}

\end{exampleblock}
\end{frame}


%-------------------------------------------------------------------------------------
\begin{frame}[fragile]
  \frametitle{Submitting jobs}
\begin{itemize}
  \item Open the batch script \verb|sbatch_beskow.sh| in your favorite editor (emacs or vim)
    \begin{itemize}
      \item Set the allocation ID to \verb|edu18.intropdc|
      \item Set that the job should run on two nodes, with 16 processes running on each node
      \item Set the requested time of the job to 2 minutes
    \end{itemize}
  \item Submit the job to the SLURM queue!
  \item Monitor the queue to see if your job is running
  \item What output did you get?
\end{itemize}
\end{frame}


%-------------------------------------------------------------------------------------
\begin{frame}[fragile]
  \frametitle{Submitting jobs}
\begin{exampleblock}{{Answer}}
    \verbatimfont{\footnotesize}
    \begin{itemize}
    \item The allocation, number of nodes and time are set with these flags
    \begin{verbatim}
      #SBATCH -A edu18.intropdc
      #SBATCH -t 0:02:00
      #SBATCH --nodes=2
    \end{verbatim}
    \item If you want to run 32 MPI processes in total, with 16 processes on each node, both the \verb|-n| and \verb|-N| flags 
      to \verb|aprun| must be used:
    \begin{verbatim}
      aprun -n 32 -N 16 ./hello_world_mpi
    \end{verbatim}
    \item The job is submitted and monitored like this:
    \begin{verbatim}
      $ sbatch sbatch_beskow.sh
      $ squeue -u <username>
    \end{verbatim}
    \item The output gets written to a default filename:
    \begin{verbatim}
      $ cat slurm-2559495.out
      Hello world from rank 0 out of 32 process
      Hello world from rank 4 out of 32 process
      ...
    \end{verbatim}

    \end{itemize}

\end{exampleblock}
\end{frame}


%-------------------------------------------------------------------------------------
\begin{frame}[fragile]
  \frametitle{Running interactively}
\begin{itemize}
  \item For this example, use the OpenMP code \verb|hello_world_openmp.c|.
    \begin{itemize}
      \item First compile it. Does it work to compile with \verb|cc -o hello_world_openmp hello_world_openmp.c|?
      \item Spoiler: it won't work, since you need to add an OpenMP compiler flag. Have a look in
        the Makefile to find the flag
    \end{itemize}
  \item Now allocate an interactive node. Hint: you need to specify allocation, number of nodes and time.
  \item Run the compiled \verb|hello_world_openmp| on the interactive node. Important: don't forget to use \verb|aprun|
  \item How many OpenMP threads do you see? How do you set the number of threads?
\end{itemize}
\end{frame}


%-------------------------------------------------------------------------------------
\begin{frame}[fragile]
  \frametitle{Running interactively}
\begin{exampleblock}{{Answer}}
    \verbatimfont{\footnotesize}
    \begin{itemize}
    \item To compile the code with OpenMP support, do:
    \begin{verbatim}
      $ cc -fopenmp -o hello_world_openmp hello_world_openmp.c
    \end{verbatim}

    \item To allocate an interactive node for 10 minutes:
    \begin{verbatim}
      $ salloc -A edu18.intropdc -t 0:05:0 -N 1
    \end{verbatim}

    \item To run the code using 32 threads, first set this environment variable and then run with \verb|aprun|
    \begin{verbatim}
      $ export OMP_NUM_THREADS=32
      $ aprun -n 1 -d 32 ./hello_world_openmp  
      # the -d (depth) isn't actually needed here
    \end{verbatim}
    NOTE: \verb|aprun| sends the job to the interactive compute node. Omitting \verb|aprun| means the calculation will run 
    on the login node.

    \end{itemize}

\end{exampleblock}
\end{frame}





%-------------------------------------------------------------------------------------


\frame{\huge\centering Questions...?}
