\section*{Login and running}

\begin{frame}[fragile]
  \frametitle{Login and running}

\footnotesize
\begin{exampleblock}{\large{Login}}
\begin{itemize}
  \item Some configuration steps are needed to log in to PDC 
  \item Depends on OS: https://www.pdc.kth.se/support/documents/login/login.html
  \item In short, Kerberos and SSH supporting GSSAPI key exchange must be installed 
  \item Everything is already configured on the lab room Ubuntu machines, \alert{provided you use these custom commands:} \verb|pdc-kinit|, \verb|pdc-klist|, \verb|pdc-ssh|, \verb|pdc-scp|, ...
  \item If needed, you will receive help to connect from your own laptops
  \item We also have bootable USB sticks with Ubuntu where everything is ready
\end{itemize}
\end{exampleblock}

\begin{exampleblock}{\large{Live demo}}
\begin{itemize}
  \item We will now demonstrate some key steps in logging in and running on Beskow
  \item Remember to replace \verb|kinit|, \verb|ssh| etc. with \verb|pdc-kinit|, \verb|pdc-ssh| etc. if working on KTH-Ubuntu machine
\end{itemize}
\end{exampleblock}

\end{frame}


%-------------------------------------------------------------------------------------
\begin{frame}[fragile]
  \frametitle{SSH}
  \begin{alertblock}{SSH configuration}
    \verbatimfont{\footnotesize}
    \begin{verbatim}
kthw@local~$ cat .ssh/config
# Hosts we want to authenticate to with Kerberos
Host *.kth.se *.kth.se.
# User authentication based on GSSAPI is allowed
GSSAPIAuthentication yes
# Key exchange based on GSSAPI may be used for server authentication
GSSAPIKeyExchange yes
# Hosts to which we want to delegate credentials
Host *.csc.kth.se *.csc.kth.se. *.nada.kth.se *.nada.kth.se. \
     *.pdc.kth.se *.pdc.kth.se.
# Forward (delegate) credentials (tickets) to the server.
GSSAPIDelegateCredentials yes
# Prefer GSSAPI key exchange
PreferredAuthentications gssapi-keyex,gssapi-with-mic
# All other hosts
Host *

 \end{verbatim}
\end{alertblock}

\end{frame}

%-------------------------------------------------------------------------------------

\begin{frame}[fragile]
  \frametitle{Kerberos}
  \begin{alertblock}{Kerberos configuration}
    \verbatimfont{\footnotesize}
    \begin{verbatim}

kthw@local~$ cat /etc/krb5.conf
[domain_realm]
   .pdc.kth.se = NADA.KTH.SE
[appdefaults]
   forwardable = yes
   forward = yes
   krb4_get_tickets = no
[libdefaults]
   default_realm = NADA.KTH.SE
   dns_lookup_realm = true
   dns_lookup_kdc = true

 \end{verbatim}
\end{alertblock}

\end{frame}


%-------------------------------------------------------------------------------------

\begin{frame}[fragile]
  \frametitle{Kerberos}
  \begin{alertblock}{Create and list tickets}
    \verbatimfont{\footnotesize}
    \begin{verbatim}
kthw@local~$ klist
klist: No credentials cache found
      
kthw@local~$ kinit -f kthw@NADA.KTH.SE
Password for kthw@NADA.KTH.SE:

kthw@local~$ klist -Tf  
Ticket cache: KCM:501
Default principal: kthw@NADA.KTH.SE

Valid starting       Expires              Service principal
08/03/2017 16:39:56  08/04/2017 16:39:50  krbtgt/NADA.KTH.SE@NADA.KTH.SE
   Flags: FIA

  \end{verbatim}
  \end{alertblock}

\end{frame}


%-------------------------------------------------------------------------------------


\begin{frame}[fragile]
  \frametitle{Login}
  \begin{alertblock}{Log in to Beskow, check ticket}
    \verbatimfont{\footnotesize}
    \begin{verbatim}

kthw@local:~$ ssh kthw@beskow.pdc.kth.se
kthw@beskow-login2:~$ klist -f

Credentials cache: FILE:/tmp/krb5cc_H26527
    Principal: kthw@NADA.KTH.SE

Issued                Expires             Flags    Principal
Aug  3 16:41:51 2017  Aug  4 16:39:50 2017  FfA    krbtgt/NADA.KTH.SE@NADA.KTH.SE
Aug  3 16:41:52 2017  Aug  4 16:39:50 2017  fA     afs/pdc.kth.se@NADA.KTH.SE
Aug  3 16:41:52 2017  Aug  4 16:39:50 2017  fA     afs@NADA.KTH.SE

 \end{verbatim}
\end{alertblock}



\end{frame}


%-------------------------------------------------------------------------------------


\begin{frame}[fragile]
  \frametitle{Modules}
  \begin{alertblock}{Inspect module system}
    \verbatimfont{\footnotesize}
    \begin{verbatim}

kthw@beskow-login2:~$ module list
...
kthw@beskow-login2:~$ module avail
...
kthw@beskow-login2:~$ module avail gcc
...
kthw@beskow-login2:~$ CC -V
Cray C++ : Version 8.3.4  Mon Aug 07, 2017  15:04:06
kthw@beskow-login2:~$ module swap PrgEnv-cray PrgEnv-gnu
kthw@beskow-login2:~$ CC --version
g++ (GCC) 4.9.1 20140716 (Cray Inc.)

 \end{verbatim}
\end{alertblock}



\end{frame}


%-------------------------------------------------------------------------------------

\begin{frame}[fragile]
  \frametitle{Interactive job on Beskow}
  \begin{alertblock}{Go to Klemming and start interactive session}
    \verbatimfont{\footnotesize}
    \begin{verbatim}

kthw@beskow-login2:~$ cd /cfs/klemming/nobackup/k/kthw/

# (command line shortened below here)
$ mkdir my_job
$ cd my_job
$ salloc -A summer-2017 --reservation=summer-2017-08-15 -N 1 -t 0:10:0
salloc: Granted job allocation 1733496

$ hostname
beskow-login2.pdc.kth.se

$ aprun -n 1 hostname
nid01610

$ exit
salloc: Relinquishing job allocation 1733497
salloc: Job allocation 1733497 has been revoked.

 \end{verbatim}
\end{alertblock}



\end{frame}


%-------------------------------------------------------------------------------------

\begin{frame}[fragile]
  \frametitle{Interactive job on Tegner}
  \begin{alertblock}{Go to Klemming and start interactive session}
    \verbatimfont{\footnotesize}
    \begin{verbatim}

kthw@tegner-login-1:~$ cd /cfs/klemming/nobackup/k/kthw/

# (command line shortened below here)
$ mkdir my_job
$ cd my_job
$ salloc -A summer-2017 -N 1 -t 0:10:0
salloc: job 666061 has been allocated resources
salloc: Waiting for resource configuration
salloc: Nodes t03n03 are ready for job

# either run on allocated node using srun/mpirun, or by logging into node
$ srun -n 1 hostname
t03n03.pdc.kth.se
# to log in to node, must open new terminal:
local-machine $ ssh kthw@t03n03.pdc.kth.se
t03n03 $ mpirun -np 24 my_code.x

 \end{verbatim}
\end{alertblock}



\end{frame}


%-------------------------------------------------------------------------------------

\begin{frame}[fragile]
  \frametitle{Batch job}
  \begin{alertblock}{Compile code and write batch script}
    \verbatimfont{\footnotesize}
    \begin{verbatim}

$ cp ~/Public/hello_world.f90 .
$ ftn -o hello_world.x hello_world.f90

$ cat <<EOF > submit.bash
#!/bin/bash -l

#SBATCH -A summer-2017
#SBATCH -J myjob
#SBATCH -t 0:10:00
#SBATCH -N 1
#SBATCH --reservation=summer-2017-08-15
#SBATCH -e error_file.e
#SBATCH -o output_file.o

aprun -n 32 ./hello_world.x > my_output_file 2>&1

EOF

 \end{verbatim}
\end{alertblock}


\end{frame}


%-------------------------------------------------------------------------------------

\begin{frame}[fragile]
  \frametitle{Batch job}
  \begin{alertblock}{Submit and monitor job}
    \verbatimfont{\footnotesize}
    \begin{verbatim}

$ sbatch submit.bash
$ squeue -u kthw
JOBID USER ACCOUNT NAME  ST REASON START_TIME TIME TIME_LEFT NODES CPUS
1735211 kthw pdc.sta myjob R None 2017-08-07T16:31:01 0:00 10:00 1 64

$ cat my_output_file
 Hello from rank           31  of           32
 Hello from rank           13  of           32
 Hello from rank           26  of           32
 Hello from rank           10  of           32
 Hello from rank           17  of           32
 Hello from rank           14  of           32
 Hello from rank            1  of           32
...


 \end{verbatim}
\end{alertblock}


\end{frame}


%-------------------------------------------------------------------------------------

