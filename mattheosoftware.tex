\section{Your code on Dardel}

\begin{comment}
\frame{
  \frametitle{How to get your code running on the Dardel CPU partition}
  \begin{itemize}
  \item Good practice: Build code with two different tool chains
    \item On Dardel use PrgEnv-cray and PrgEnv-gnu
    %\item Building with PrgEnv-cray toolchain
    \item Cray Compiling Environment (CCE)
    \begin{itemize}
    %\item Cray Compiling Environment (CCE)
      \item The CCE C/C++ compiler is based on Clang/LLVM.
      \item CCE Fortran Supports most of Fortran 2018 (ISO/IEC 1539:2018) with some exceptions
    \end{itemize}
  \item Building with PrgEnv-gnu toolchain
  \item Scaling analyzes and performance tuning
  \end{itemize}
}
\end{comment}

\frame{
  \frametitle{How to get your code running on the Dardel CPU partition}
  \begin{itemize}
    \item Good practice: Build code with two different tool chains
    \item On Dardel use PrgEnv-cray and PrgEnv-gnu
    \item For libraries and include files covered by module files, you should not add anything to your Makefile
    \begin{itemize}
      \item No additional MPI flags are needed (included by wrappers)
      \item No need to add any -I, -l or –L flags for the Cray provided libraries
      \item If Makefile needs an input for –L work correctly, try using '.'
    \end{itemize}
    \item OpenMP is supported by all of the PrgEnvs
    \begin{itemize}
      \item PrgEnv-cray (Fortran) -homp
      \item PrgEnv-cray (C/C++) -fopenmp
      \item PrgEnv-gnu -fopenmp
    \end{itemize}
  \end{itemize}
}

\begin{comment}
\frame{
  \frametitle{PrgEnv-cray
  \begin{itemize}
    \item The PrgEnv-cray is the native environment on a HPE Cray EX supercomputer
    \item Cray Compiling Environment (CCE)
    \begin{itemize}
      \item The CCE C/C++ compiler is based on Clang/LLVM.
      \item CCE Fortran Supports most of Fortran 2018 (ISO/IEC 1539:2018) with some exceptions
    \end{itemize}
    \item Cray Scientific and Math Libraries (CSML)
    \begin{itemize}
        \item High Performance libraries for scientific applications (BLAS, LAPACK, ScaLAPCK, FFTW, NetCDF)
    \end{itemize}
    \item Cray Message Passing Toolkit (CMPT)
    \begin{itemize}
      \item Provides the Message Passing Interface (MPI) and OpenSHMEM
    \end{itemize}
    %\item Cray Environment Setup and Compiling Support (CENV)
    %\item Infrastructure to support the development environment
    %\item Includes compiler drivers, hugepages support and the PE packaging API
    %\item Cray Performance Measurement and Analysis Tools (CPMAT)
    %\item Provides tools to analyse performance and behaviour of programs and the PAPI performance API
    %\item Cray Debugging Support Tools (CDST)
    %\item Provides debugging tools including gdb4hpc and valgrind4hpc
  \end{itemize}
}
\end{comment}

\frame{
  \frametitle{How to get your code running on the Dardel GPU partition}
  \begin{itemize}
  \item Porting of CUDA code to HIP with hipify
  \item Higher level offloading with openMP
  \item Lower level offloading to GPU with HIP
  %\item \emph{list other routes to get code on AMD GPUs}
  \end{itemize}
}

\frame{
  \frametitle{Needs and challenges for specific codes}
  \framesubtitle{Uppsala / KTH / Örebro / Belém / Pohang materials theory codes}
  \begin{itemize}
  \item RSPt
  \item Elk
  \item UppASD
  \item TRIQS/cthyb
  \item RS-LMTO-ASA
  \end{itemize}
  \begin{itemize}
  \item RSPt
  \end{itemize}
}
