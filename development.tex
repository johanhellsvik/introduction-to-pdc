\section{Development}

\subsection{Modules}
\begin{frame}[fragile]
\frametitle{Modules}
\framesubtitle{Using Lmod}

\begin{exampleblock}{List loaded modules}
  \begin{verbatim}
  ml
  \end{verbatim}
\end{exampleblock}

\begin{exampleblock}{List available modules}
  \begin{verbatim}
  ml avail
  \end{verbatim}
\end{exampleblock}

\begin{exampleblock}{Load modules}
  \begin{verbatim}
  ml <software_name>
  \end{verbatim}
\end{exampleblock}

\begin{exampleblock}{Unload modules}
  \begin{verbatim}
  ml -<software_name>
  \end{verbatim}
\end{exampleblock}
\end{frame}


\begin{frame}[fragile]
\frametitle{Modules}
\framesubtitle{Displaying modules}
\begin{exampleblock}{\$ ml}
\scriptsize
\begin{verbatim}
Currently Loaded Modulefiles:
  1) craype-x86-rome
  ...
  10) cray-libsci/21.08.1.2
\end{verbatim}
\end{exampleblock}

\begin{exampleblock}{\$ ml avail [software\_name]}
\scriptsize
\begin{verbatim}
---------------- /opt/cray/pe/lmod/modulefiles/cpu/x86-rome/1.0 ---------------
     cray-fftw/3.3.8.10    cray-fftw/3.3.8.11    cray-fftw/3.3.8.12 (D)
\end{verbatim}
\end{exampleblock}

\begin{exampleblock}{\$ module show [software\_name]}
\scriptsize
\begin{verbatim}
...
whatis("FFTW 3.3.8.12 - Fastest Fourier Transform in the West")
setenv("FFTW_VERSION","3.3.8.12")
setenv("CRAY_FFTW_VERSION","3.3.8.12")
setenv("FFTW_ROOT","/opt/cray/pe/fftw/3.3.8.12/x86_rome")
...
\end{verbatim}
\end{exampleblock}
\end{frame}


\begin{frame}[fragile]
\frametitle{Modules}
\framesubtitle{Using PDC module}
\begin{exampleblock}{
The PDC module enables many PDC-installed software modules.
}
\footnotesize
\begin{verbatim}
$ ml PDC
$ ml avail
------------- /pdc/software/21.11/other/modules -----------------
   EasyBuild-production/4.5.0    arm/21.1           fluent/21.2   ...
   ...

------------- /pdc/software/21.11/eb/modules/all ----------------
   ABINIT/9.6.2-cpeGNU-21.11         GROMACS/2021.3-cpeCray-21.11 ...
   ...

------------- /pdc/software/21.11/spack/modules -----------------
   all-spack-modules/0.17.0    amdlibm/3.0         gnuplot/5.4.2  ...
   ...

\end{verbatim}
\end{exampleblock}
\end{frame}



\begin{frame}[fragile]
\frametitle{Modules}
\framesubtitle{Using common software}
\begin{exampleblock}{Find the modules you need}
\footnotesize
\begin{verbatim}
$ ml PDC
$ ml avail gromacs
--------------------- /pdc/software/21.11/eb/modules/all ---------------
   GROMACS/2021.3-cpeCray-21.11
$ ml avail vasp
--------------------- /pdc/software/21.11/other/modules ----------------
   vasp/5.4.4-vanilla    vasp/5.4.4-wannier90    vasp/6.2.1-vanilla  ...
$ ml avail fftw
-------------- /opt/cray/pe/lmod/modulefiles/cpu/x86-rome/1.0 ----------
   cray-fftw/3.3.8.10    cray-fftw/3.3.8.11    cray-fftw/3.3.8.12 (D)
\end{verbatim}
\end{exampleblock}

Example submission scripts for common software can be found in:
\href{https://www.pdc.kth.se/software}{https://www.pdc.kth.se/software}
\end{frame}


\begin{frame}[fragile]
\frametitle{Modules}
\framesubtitle{Using singularity}
\begin{exampleblock}{Singularity}
\begin{itemize}
\item Open-source container system for HPC
\item Brings portability and reproducibility
\end{itemize}
\end{exampleblock}
\begin{exampleblock}{To use Singularity}
\begin{itemize}
\item Get your singlarity image
  \begin{itemize}
  \item Download images from singularity hub, or
  \item Build your own image (on your own computer)
  \end{itemize}
\item Run singularity image on Dardel
\end{itemize}
\end{exampleblock}
\scriptsize
    \href{https://www.pdc.kth.se/software/software/singularity/cpe21.09/3.8.3-1/index\_using.html}{https://www.pdc.kth.se/software/software/singularity/cpe21.09/3.8.3-1/index\_using.html}
\end{frame}


\subsection{Programming environments}

\begin{frame}[fragile]
\frametitle{Programming Environment Modules}
\begin{exampleblock}{Programming Environment on Dardel}
\begin{itemize}
  \item PrgEnv-cray:
    loads the Cray compiling environment (CCE) that provides compilers for Cray systems.
  \item PrgEnv-gnu:
    loads the GNU compiler suite.
  \item PrgEnv-aocc:
    loads the AMD AOCC compilers.
\end{itemize}
\end{exampleblock}
\begin{columns}[t]
\column{1.\textwidth}
\begin{description}
    \item [Cray] \verb|$ ml PrgEnv-cray|
    \item [GNU]  \verb|$ ml PrgEnv-gnu|
    \item [AMD]  \verb|$ ml PrgEnv-aocc|
\end{description}
 % \item Module cray-libsci provides BLAS, LAPACK, BLACS, and SCALAPACK
 % \item Module cray-mpich provides MPI
\end{columns}
\end{frame}


\begin{frame}[fragile]
\frametitle{Programming Environment Modules}
    \begin{exampleblock}{ Use cpe module with PrgEnv- modules }
    \footnotesize
    \begin{verbatim}
$ ml PrgEnv-gnu

Lmod is automatically replacing "cce/13.0.0" with "gcc/11.2.0".

Lmod is automatically replacing "PrgEnv-cray/8.2.0" with 
"PrgEnv-gnu/8.2.0".

Due to MODULEPATH changes, the following have been reloaded:
  1) cray-mpich/8.1.11

$ ml cpe

$ cc --version
gcc (GCC) 11.2.0 20210728 (Cray Inc.)
Copyright (C) 2021 Free Software Foundation, Inc.
...
    \end{verbatim}
    \end{exampleblock}
\end{frame}


\subsection{Compiling code}
\begin{frame}[fragile]
\frametitle{Compiling, Linking and Running Applications}
\framesubtitle{on HPC clusters}
 \begin{description}
    \item [source code] C / C++ / Fortran ( \verb|.c, .cpp, .f90, .h|  )
    \item [compile] Cray/GNU/AMD compilers
    \item [assemble] into machine code (object files: \verb|.o, .obj| )
    \item [link] Static Libraries (\verb|.lib, .a|  ) \\ Shared Library (\verb|.dll, .so| ) \\ Executables (\verb|.exe, .x| )
    \item ~ 
    \item [request allocation] submit job request to SLURM queuing system \\ \verb|salloc/sbatch|
    \item [run] application on scheduled resources \\ \verb|srun|
 \end{description}
\end{frame}

\begin{frame}[fragile]
\frametitle{Compiler wrappers}
\framesubtitle{cc, CC and ftn}
\begin{columns}[t]
\column{1.\textwidth}
  \begin{description}
      \item [C] \verb|$ cc -o myexe.x mycode.c|
      \item [C++] \verb|$ CC -o myexe.x mycode.cpp|
      \item [Fortran] \verb|$ ftn -o myexe.x mycode.f90|
  \end{description}
   % \item Module cray-libsci provides BLAS, LAPACK, BLACS, and SCALAPACK
   % \item Module cray-mpich provides MPI
\end{columns}
  \begin{exampleblock}{Compiler wrappers : \alert{\textbf{cc} \textbf{CC} \textbf{ftn}}}
    \alert{Advantages}\\
    Compiler wrappers will automatically 
    \begin{itemize}
      \item link to BLAS, LAPACK, BLACS, SCALAPACK, FFTW\\
      \item link to MPI\\
    \end{itemize}
    \alert{Disadvantage}\\
    Sometimes you need to edit Makefiles which are not designed for Cray 
\end{exampleblock}
\end{frame}
